\documentclass[12pt]{article}
\usepackage[backend=biber]{biblatex}
\addbibresource{ffhp2.bib}
\pagestyle{empty}
\textwidth      165mm
\textheight     240mm
\topmargin      -18mm
\oddsidemargin  -2mm
\evensidemargin 2mm

\author{Emmanuel Macario - 831659\\\\Word Count: 1,035}
\title{UNIB20014 Food for a Healthy Planet II \\ Assignment One\\ Article 2 - Critical Research Essay (Critique)}
\date{25 September, 2018}

\begin{document}

\maketitle

\pagebreak
“I quit sugar 3 (why sugar makes us fat)” is an article written by Australian blogger Sarah Wilson and was published on February 4, 2011 on her personal website www.sarahwilson.com. Wilson contends that in comparison to other macromolecules such as fat and protein, it is mainly carbohydrates that are responsible for adverse health effects including weight gain. To persuade the reader that “fat doesn’t make us fat, sugar does”, Wilson purports that fructose, a key component of sugar, is responsible for fattening due to its inability to satiate the human body. This notion forms the backbone of the author’s argument, allowing her to proclaim the positive health benefits of removing sugar from the diet, and the need to substitute sugary foods with protein and fat-rich foods.

\bigskip
However, since Wilson is the author of the best-selling book “I Quit Sugar”, the objectivity of her claims must be put into question. A question that must be asked, is whether the author is promulgating these statements to promote her own book and personal agenda, or if there is some merit to these claims. The purpose of this essay is to critique the article and analyse it rigorously with scientific evidence, to determine the veracity and truthfulness of Wilson’s assertions.

\bigskip
The article is structured in a manner such that there is a heavy usage of dot points, with each including some form of evidence supporting the negative consequences of sugar consumption. Initially, the article begins with a brief overview, highlighting the author’s perspective on sugar by using anecdotal evidence that protein and fat satiates, while sugar does not. Subsequently, Wilson employs the use of four succinct bolded statements, either a claim or a rhetoric. Under each of these headings follows a plethora of evidence points that supports each claim. The article flows easily due to the nature of its structure, since each point is directly followed by pieces of evidence. Due to this, it may make it difficult for the reader to question the authenticity of the claims. It is also worth noting the deficiency of references to supporting publications, or rebuttals from opposing viewpoints, except for one reference to David Gillespie’s book “Sweet Poison”.

\bigskip
Wilson argues that due to humans inherently having a “biological predisposition to seek out sugar”, humans tend to overemphasise its nutritional value, leading to a level of overconsumption that allows sugar to exert its deleterious effects. \textcite{tappy} found concurring evidence in their experiments that revealed when dietary fructose is consumed excessively, it may have adverse metabolic effects, such as hepatic insulin resistance and increased fat storage. Conversely, \textcite{markus} falsifies the claim that sugar is the sole culprit for weight gain, in a larger cross-sectional study of 1,495 university students’ eating dependencies and weight. In this meta-analysis, overweight was very highly correlated with high-fat savoury and sweet foods, with only a small correlation between high-sugar concentration foods and overweight. The findings of both reports stipulate that while sugary foods contribute to food dependence and weight gain, it does so in a moderate fashion, with excess consumption of fat and protein still being a major contributor to weight gain. This disputes the author’s claim that “we don’t get fat from eating fat and protein”. Rather, these findings support the scientific notion that food-energy density, and that an individual’s unique experience of food consumption, plays an important role in determining the role of food in excessive energy intake and subsequent weight gain.

\bigskip
In an attempt to appear impartial, Wilson additionally acknowledges the potential of fructose to be beneficial to the human body because it does not cause insulin spikes. However, she refutes this claim by stating that this is untrue for “a host of reasons”, but only provides one single reason for this. The difficulty with this claim is that it is unsupported by factual evidence for the reader to discern and make their own informed judgement. The author also asserts that fructose is fattening since it is converted directly to fatty acids and then bodily fat, via metabolic processes in the liver. This claim is upheld by recent research regarding fructose metabolism \textcite{tappy}, in which fructose is revealed to be mainly processed in the splanchnic organs including the gut, liver and kidneys, to glucose and fatty acids, which can then be oxidised in organs and tissues. Hence, it can be seen that the author’s statement that consumption of fructose can directly lead to weight gain does have some merit. However, opposers of this sentiment believe that the adverse metabolic effects of fructose, including increased fat deposition and insulin resistance, are undermined \textcite{macdonald}. This is because much of the evidence is based off animal studies and not human studies, which provide a different metabolic picture.

\bigskip
The urgency of the Wilson’s contention is made apparent as she concludes in an assertive manner that “fructose is to be avoided at all costs”. However, this subtle plea goes against her directly previous statement stating that miniscule quantities of fructose are acceptable for normal consumption, without the consequence of weight gain. This notion is backed by an evidence-based study, in which \textcite{dolan} demonstrates that there is no significant relationship between the consumption of fructose in a normal dietary manner and an increase in weight in both normal weight and overweight individuals. Furthermore, results indicated that standard levels of fructose consumption do not cause biologically relevant changes in blood triglyceride levels. This conflicting revelation dismisses Wilson’s claim that only “when we’re in balance, and eating no sugar, our bodies don’t put on weight”. Consequently, Wilson’s claim that ingestion of fructose is solely responsible for fattening is falsified, as it is clear that standard levels of fructose consumption may not lead to weight gain.

\bigskip
Conclusively, Wilson’s claims display a reasonable level of merit. In hindsight, the article may have been more convincing if it employed the use of more qualitative statistics, and references to where the existing statistics and facts in the evidence originate from. Through proper analysis and research, it can be seen that some potential falsities may exist in the article. These are most notable when Wilson fervently proclaims definitive statements that are not evidently supported by scientific research and findings.


\pagebreak
\printbibliography
\end{document}